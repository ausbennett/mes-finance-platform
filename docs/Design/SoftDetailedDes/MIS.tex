\documentclass[12pt, titlepage]{article}

\usepackage{amsmath, mathtools}

\usepackage[round]{natbib}
\usepackage{amsfonts}
\usepackage{amssymb}
\usepackage{graphicx}
\usepackage{colortbl}
\usepackage{xr}
\usepackage{hyperref}
\usepackage{longtable}
\usepackage{xfrac}
\usepackage{tabularx}
\usepackage{float}
\usepackage{siunitx}
\usepackage{booktabs}
\usepackage{multirow}
\usepackage[section]{placeins}
\usepackage{caption}
\usepackage{fullpage}

\hypersetup{
bookmarks=true,     % show bookmarks bar?
colorlinks=true,       % false: boxed links; true: colored links
linkcolor=red,          % color of internal links (change box color with linkbordercolor)
citecolor=blue,      % color of links to bibliography
filecolor=magenta,  % color of file links
urlcolor=cyan          % color of external links
}

\usepackage{array}

\externaldocument{../../SRS/SRS}

%% Comments

\usepackage{color}

\newif\ifcomments\commentstrue %displays comments
%\newif\ifcomments\commentsfalse %so that comments do not display

\ifcomments
\newcommand{\authornote}[3]{\textcolor{#1}{[#3 ---#2]}}
\newcommand{\todo}[1]{\textcolor{red}{[TODO: #1]}}
\else
\newcommand{\authornote}[3]{}
\newcommand{\todo}[1]{}
\fi

\newcommand{\wss}[1]{\authornote{blue}{SS}{#1}} 
\newcommand{\plt}[1]{\authornote{magenta}{TPLT}{#1}} %For explanation of the template
\newcommand{\an}[1]{\authornote{cyan}{Author}{#1}}

%% Common Parts

\newcommand{\progname}{McMaster Engineering Society Custom Financial
Expense Reporting Platform} % PUT YOUR PROGRAM NAME HERE
\newcommand{\authname}{Team \#12, Reimbursement Rangers
\\ Adam Podolak
\\ Evan Sturmey
\\ Christian Petricca
\\ Austin Bennett
\\ Jacob Kish} % AUTHOR NAMES                  

\usepackage{hyperref}
    \hypersetup{colorlinks=true, linkcolor=blue, citecolor=blue, filecolor=blue,
                urlcolor=blue, unicode=false}
    \urlstyle{same}
                                


\begin{document}

\title{Module Interface Specification for \progname{}}

\author{\authname}

\date{\today}

\maketitle

\pagenumbering{roman}

\section{Revision History}

\begin{tabularx}{\textwidth}{p{3cm}p{2cm}X}
\toprule {\bf Date} & {\bf Version} & {\bf Notes}\\
\midrule
Date 1 & 1.0 & Notes\\
Date 2 & 1.1 & Notes\\
\bottomrule
\end{tabularx}

~\newpage

\section{Symbols, Abbreviations and Acronyms}

See SRS Documentation at \wss{give url}

\wss{Also add any additional symbols, abbreviations or acronyms}

\newpage

\tableofcontents

\newpage

\pagenumbering{arabic}

\section{Introduction}

The following document details the Module Interface Specifications for
the MES Finance Platform. The document specifies how each module interfaces with other parts of the program. Complementary documents include the System Requirement Specifications and Module Guide.  The full documentation and implementation can be found at \url{https://github.com/ausbennett/mes-finance-platform}.

\section{Notation}

\wss{You should describe your notation.  You can use what is below as
  a starting point.}

The structure of the MIS for modules comes from \citet{HoffmanAndStrooper1995},
with the addition that template modules have been adapted from
\cite{GhezziEtAl2003}.  The mathematical notation comes from Chapter 3 of
\citet{HoffmanAndStrooper1995}.  For instance, the symbol := is used for a
multiple assignment statement and conditional rules follow the form $(c_1
\Rightarrow r_1 | c_2 \Rightarrow r_2 | ... | c_n \Rightarrow r_n )$.

The following table summarizes the primitive data types used by \progname. 

\begin{center}
\renewcommand{\arraystretch}{1.2}
\noindent 
\begin{tabular}{l l p{7.5cm}} 
\toprule 
\textbf{Data Type} & \textbf{Notation} & \textbf{Description}\\ 
\midrule
character & char & a single symbol or digit\\
integer & $\mathbb{Z}$ & a number without a fractional component in (-$\infty$, $\infty$) \\
natural number & $\mathbb{N}$ & a number without a fractional component in [1, $\infty$) \\
real & $\mathbb{R}$ & any number in (-$\infty$, $\infty$)\\
\bottomrule
\end{tabular} 
\end{center}

\noindent
The specification of \progname \ uses some derived data types: sequences, strings, and
tuples. Sequences are lists filled with elements of the same data type. Strings
are sequences of characters. Tuples contain a list of values, potentially of
different types. In addition, \progname \ uses functions, which
are defined by the data types of their inputs and outputs. Local functions are
described by giving their type signature followed by their specification.

\section{Module Decomposition}

The following table is taken directly from the Module Guide document for this project.

\begin{table}[h!]
\centering
\begin{tabular}{p{0.3\textwidth} p{0.6\textwidth}}
\toprule
\textbf{Level 1} & \textbf{Level 2}\\
\midrule

{Hardware-Hiding} & ~ \\
\midrule

\multirow{7}{0.3\textwidth}{Behaviour-Hiding} 
& Account Management Module\\
& Requests Module\\
& Notification Module\\
& User Dashboard Module\\
& Authentication Module\\ 
& Email Module\\
& Account Management Controller Module\\
& Requests Controller Module\\
\midrule

\multirow{3}{0.3\textwidth}{Software Decision} 
& Clubs Database\\
& Users Database\\
& Requests Database\\
& Graphical User Interface\\
\bottomrule

\end{tabular}
\caption{Module Hierarchy}
\label{TblMH}
\end{table}

\newpage
~\newpage

\section{MIS of \wss{Module Name}} \label{Module} \wss{Use labels for
  cross-referencing}

\wss{You can reference SRS labels, such as R\ref{R_Inputs}.}

\wss{It is also possible to use \LaTeX for hypperlinks to external documents.}

\subsection{Module}

\wss{Short name for the module}

\subsection{Uses}


\subsection{Syntax}

\subsubsection{Exported Constants}

\subsubsection{Exported Access Programs}

\begin{center}
\begin{tabular}{p{2cm} p{4cm} p{4cm} p{2cm}}
\hline
\textbf{Name} & \textbf{In} & \textbf{Out} & \textbf{Exceptions} \\
\hline
\wss{accessProg} & - & - & - \\
\hline
\end{tabular}
\end{center}

\subsection{Semantics}

\subsubsection{State Variables}

\wss{Not all modules will have state variables.  State variables give the module
  a memory.}

\subsubsection{Environment Variables}

\wss{This section is not necessary for all modules.  Its purpose is to capture
  when the module has external interaction with the environment, such as for a
  device driver, screen interface, keyboard, file, etc.}

\subsubsection{Assumptions}

\wss{Try to minimize assumptions and anticipate programmer errors via
  exceptions, but for practical purposes assumptions are sometimes appropriate.}

\subsubsection{Access Routine Semantics}

\noindent \wss{accessProg}():
\begin{itemize}
\item transition: \wss{if appropriate} 
\item output: \wss{if appropriate} 
\item exception: \wss{if appropriate} 
\end{itemize}

\wss{A module without environment variables or state variables is unlikely to
  have a state transition.  In this case a state transition can only occur if
  the module is changing the state of another module.}

\wss{Modules rarely have both a transition and an output.  In most cases you
  will have one or the other.}

\subsubsection{Local Functions}

\wss{As appropriate} \wss{These functions are for the purpose of specification.
  They are not necessarily something that is going to be implemented
  explicitly.  Even if they are implemented, they are not exported; they only
  have local scope.}

\section{MIS of Account Management API} \label{AccountManagementAPI}

\subsection{Module}
Account Management API

\subsection{Uses}
Account Management Controller

\subsection{Syntax}

\subsubsection{Exported Constants}

\subsubsection{Exported Access Programs}

\begin{center}
\begin{tabular}{p{2cm} p{4cm} p{2cm} p{4cm}}
\hline
\textbf{Name} & \textbf{In} & \textbf{Out} & \textbf{Exceptions} \\
\hline
auth & token: String & JSON & InvalidTokenException \\
loginUser & email: String & String & EmailNotFoundException \\
registerUser & userDetails: JSON & JSON & DatabaseException \\
getAllUsers & adminToken: String & JSON & AuthorizationException \\
getUser & userId: String & JSON & UserNotFoundException \\
editProfile & userId: String, updates: JSON & Boolean & DatabaseException \\
editClub & clubId: String, updates: JSON & JSON & AuthorizationException \\
\hline
\end{tabular}
\end{center}

\subsection{Semantics}

\subsubsection{State Variables}
None

\subsubsection{Environment Variables}
MongoDB connection (via Mongoose)

\subsubsection{Assumptions}
\begin{itemize}
  \item Valid and authenticated tokens are provided for admin and user-specific actions.
  \item All inputs are sanitized before being processed.
\end{itemize}

\subsubsection{Access Routine Semantics}

\noindent auth(token: String):
\begin{itemize}
  \item transition: Validates the provided token and grants access.
  \item output: returns a JSON object with detailed information about the result.
  \item exception: InvalidTokenException if token is malformed or expired.
\end{itemize}

\noindent loginUser(email: String):
\begin{itemize}
  \item transition: Sends a confirmation link to the provided email.
  \item output: returns a JSON object with detailed information about the result.
  \item exception: EmailNotFoundException if email does not exist in the system.
\end{itemize}

\noindent registerUser(userDetails: JSON):
\begin{itemize}
  \item transition: Adds a new user record to the database.
  \item output: returns a JSON object with detailed information about the result.
  \item exception: DatabaseException if there is an issue saving to MongoDB.
\end{itemize}

\noindent getAllUsers(adminToken: String):
\begin{itemize}
  \item input: admin auth token 
  \item output: returns a JSON object with detailed information about the result and array (users).
  \item exception: DatabaseException if there is an issue communicating to MongoDB.
\end{itemize}

\noindent getUser(userID: String):
\begin{itemize}
  \item input: userID of user
  \item output: returns a JSON object with detailed information about the result.
  \item exception: DatabaseException if there is an issue communicating to MongoDB.
\end{itemize}

\noindent editUser(userID: String, updates: JSON):
\begin{itemize}
  \item input: userID, and a JSON object containing updates to user information.
  \item output: returns a JSON object with detailed information about the result.
  \item exception: DatabaseException if there is an issue communicating to MongoDB.
\end{itemize}

\noindent editClub(clubID: String, updates: JSON):
\begin{itemize}
  \item input: clubID, and a JSON object containing updates to club information.
  \item output: returns a JSON object with detailed information about the result.
  \item exception: DatabaseException if there is an issue communicating to MongoDB.
\end{itemize}

\subsubsection{Local Functions}
None

\section{MIS of Requests Module} \label{RequestsModule}

\subsection{Module}
Requests

\subsection{Uses}
Requests Controller, Plaid Service API

\subsection{Syntax}

\subsubsection{Exported Constants}
None

\subsubsection{Exported Access Programs}
\begin{center}
\begin{tabular}{p{2cm} p{4cm} p{4cm} p{2cm}}
\hline
\textbf{Name} & \textbf{In} & \textbf{Out} & \textbf{Exceptions} \\
\hline
submitReimbursement & requestData: JSON & Boolean & ValidationException \\
submitPayment & paymentData: JSON & Boolean & PaymentProcessingException \\
processLedger & ledgerData: JSON & Boolean & ReconciliationException \\
\hline
\end{tabular}
\end{center}

\subsection{Semantics}

\subsubsection{State Variables}
None

\subsubsection{Environment Variables}
- Plaid Service API for payment and ledger reconciliation

\subsubsection{Assumptions}
- All inputs are validated prior to processing.
- Plaid Service API is available and operational.

\subsubsection{Access Routine Semantics}
\noindent submitReimbursement(requestData: JSON):
\begin{itemize}
\item transition: Stores the reimbursement request and initiates processing via the Requests Controller.
\item output: Returns true if the request is successfully submitted.
\item exception: ValidationException if the input data is invalid.
\end{itemize}

\subsubsection{Local Functions}
None
\section{MIS of Notification Module} \label{NotificationModule}

\subsection{Module}
Notification Module

\subsection{Uses}
Requests Module

\subsection{Syntax}

\subsubsection{Exported Constants}
None

\subsubsection{Exported Access Programs}

\begin{center}
\begin{tabular}{p{2cm} p{4cm} p{2cm} p{4cm}}
\hline
\textbf{Name} & \textbf{In} & \textbf{Out} & \textbf{Exceptions} \\
\hline
notifyUser & email: String & String & EmailNotFoundException \\
\hline
\end{tabular}
\end{center}

\subsection{Semantics}

\subsubsection{State Variables}
User Details (email and notification status)

\subsubsection{Environment Variables}
None

\subsubsection{Assumptions}
\begin{itemize}
\end{itemize}

\subsubsection{Access Routine Semantics}

\noindent notifyUser(email: String):
\begin{itemize}
  \item transition: Queries Requests module for user info including email and request status.
  \item output: Returns an email body to be given to emailer API.
  \item exception: EmailNotFoundException if the user has no valid email to be returned.
\end{itemize}

\subsubsection{Local Functions}
\begin{itemize}
  \item Validation functions for email.
  \item Functions to compose email body.
\end{itemize}

\section{MIS of User Dashboard Module} \label{UserDashboard} \subsection{Module} User Dashboard

\subsection{Uses}
Requests Module, Account Management API, Authentication Module

\subsection{Syntax}

\subsubsection{Exported Constants}
None

\subsubsection{Exported Access Programs}
\begin{center}
\begin{tabular}{p{2cm} p{4cm} p{4cm} p{2cm}}
\hline
\textbf{Name} & \textbf{In} & \textbf{Out} & \textbf{Exceptions} \\
\hline
viewDashboard & userId: String & JSON & AuthorizationException \\
viewRequests & userId: String & Array (Requests) & AuthorizationException \\
editProfile & userId: String, updates: JSON & Boolean & UpdateException \\
\hline
\end{tabular}
\end{center}

\subsection{Semantics}

\subsubsection{State Variables}
None

\subsubsection{Environment Variables}
- Connections to other modules for data abstraction.

\subsubsection{Assumptions}
- The user is authenticated and authorized before accessing the dashboard.

\subsubsection{Access Routine Semantics}
\noindent viewDashboard(userId: String):
\begin{itemize}
\item transition: None
\item output: Returns the user's dashboard data.
\item exception: AuthorizationException if the user is not authorized.
\end{itemize}

\subsubsection{Local Functions}
None
\section{MIS of Emailer API} \label{EmailerAPI}

\subsection{Module}
Emailer API

\subsection{Uses}
Account Management Module, Notification Module

\subsection{Syntax}

\subsubsection{Exported Constants}
Email sending address (An automated, do-not-reply email adress)
\subsubsection{Exported Access Programs}

\begin{center}
\begin{tabular}{p{2cm} p{4cm} p{2cm} p{4cm}}
\hline
\textbf{Name} & \textbf{In} & \textbf{Out} & \textbf{Exceptions} \\
\hline
sendEmail & body: JSON, address: String & Boolean & InvalidEmailException \\
\hline
\end{tabular}
\end{center}

\subsection{Semantics}

\subsubsection{State Variables}
None

\subsubsection{Environment Variables}
Connection to donotreply automated email service

\subsubsection{Assumptions}
\begin{itemize}
  \item An external API will be used. Specifics TBD
\end{itemize}

\subsubsection{Access Routine Semantics}

\noindent sendEmail(body: JSON):
\begin{itemize}
  \item transition: Sends an email with body to the address specified.
  \item output: Returns a success or failure message depending on if the email was successfully sent.
  \item exception: InvalidEmailException if the address is invalid or the body is unsendable.
\end{itemize}

\subsubsection{Local Functions}
None

\section{MIS of Account Management Controller} \label{AccountManagementController}

\subsection{Module}
Account Management Controller

\subsection{Uses}
Mongoose Schema, MongoDB

\subsection{Syntax}

\subsubsection{Exported Constants}
None

\subsubsection{Exported Access Programs}

\begin{center}
\begin{tabular}{p{2cm} p{4cm} p{2cm} p{4cm}}
\hline
\textbf{Name} & \textbf{In} & \textbf{Out} & \textbf{Exceptions} \\
\hline
createUser & userDetails: JSON & JSON & DatabaseException \\
findUser & userId: String & JSON & UserNotFoundException \\
updateUser & userId: String, updates: JSON & JSON & DatabaseException \\
deleteUser & userId: String & JSON & AuthorizationException \\
\hline
\end{tabular}
\end{center}

\subsection{Semantics}

\subsubsection{State Variables}
MongoDB User Schema (defines fields like email, password, roles, etc.)

\subsubsection{Environment Variables}
MongoDB connection via Mongoose (database connection client)

\subsubsection{Assumptions}
\begin{itemize}
  \item Mongoose (database connection client) is properly configured and connected to MongoDB.
  \item User schema validations are performed automatically during operations.
\end{itemize}

\subsubsection{Access Routine Semantics}

\noindent createUser(userDetails: JSON):
\begin{itemize}
  \item transition: Saves a new user record to MongoDB.
  \item output: Returns a JSON object with detailed information about the result.
  \item exception: DatabaseException if saving fails due to validation or connection issues.
\end{itemize}

\noindent findUser(userId: String):
\begin{itemize}
  \item transition: Queries the MongoDB collection for the specified user.
  \item output: Returns user data in JSON format.
  \item exception: UserNotFoundException if the user ID does not exist.
\end{itemize}

\noindent updateUser(userId: String, updates: JSON):
\begin{itemize}
  \item transition: Updates the MongoDB collection for the specified user information.
  \item output: Returns a JSON object with detailed information about the result.
  \item exception: UserNotFoundException if the user ID does not exist.
\end{itemize}

\noindent deleteUser(userId: String):
\begin{itemize}
  \item transition: Removes the specified user from the MongoDB collection.
  \item output: Returns a JSON object with detailed information about the result.
  \item exception: UserNotFoundException if the user ID does not exist.
\end{itemize}

\subsubsection{Local Functions}
\begin{itemize}
  \item Validation functions for email and password.
\end{itemize}

\newpage

\bibliographystyle {plainnat}
\bibliography {../../../refs/References}

\newpage

\section{MIS of Requests Controller Module} \label{RequestsController}

\subsection{Module}
Requests Controller

\subsection{Uses}
Database (via ORM)

\subsection{Syntax}

\subsubsection{Exported Constants}
None

\subsubsection{Exported Access Programs}
\begin{center}
\begin{tabular}{p{2cm} p{4cm} p{4cm} p{2cm}}
\hline
\textbf{Name} & \textbf{In} & \textbf{Out} & \textbf{Exceptions} \\
\hline
storeReimbursement & requestData: JSON & Boolean & DatabaseException \\
storePayment & paymentData: JSON & Boolean & DatabaseException \\
reconcileLedger & ledgerData: JSON & Boolean & DatabaseException \\
\hline
\end{tabular}
\end{center}

\subsection{Semantics}

\subsubsection{State Variables}
- MongoDB collections for requests and ledgers

\subsubsection{Environment Variables}
- Database connection (via Mongoose ORM)

\subsubsection{Assumptions}
- Database schema is correctly defined and applied.
- Database connection is persistent.

\subsubsection{Access Routine Semantics}
\noindent storeReimbursement(requestData: JSON):
\begin{itemize}
\item transition: Saves the reimbursement request in the database.
\item output: Returns true if the operation is successful.
\item exception: DatabaseException if the request cannot be stored.
\end{itemize}

\subsubsection{Local Functions}
None
\section{Appendix} \label{Appendix}

\wss{Extra information if required}

\newpage{}

\section*{Appendix --- Reflection}

\wss{Not required for CAS 741 projects}

The information in this section will be used to evaluate the team members on the
graduate attribute of Problem Analysis and Design.

The purpose of reflection questions is to give you a chance to assess your own
learning and that of your group as a whole, and to find ways to improve in the
future. Reflection is an important part of the learning process.  Reflection is
also an essential component of a successful software development process.  

Reflections are most interesting and useful when they're honest, even if the
stories they tell are imperfect. You will be marked based on your depth of
thought and analysis, and not based on the content of the reflections
themselves. Thus, for full marks we encourage you to answer openly and honestly
and to avoid simply writing ``what you think the evaluator wants to hear.''

Please answer the following questions.  Some questions can be answered on the
team level, but where appropriate, each team member should write their own
response:


\begin{enumerate}
  \item What went well while writing this deliverable? 
  \item What pain points did you experience during this deliverable, and how
    did you resolve them?
  \item Which of your design decisions stemmed from speaking to your client(s)
  or a proxy (e.g. your peers, stakeholders, potential users)? For those that
  were not, why, and where did they come from?
  \item While creating the design doc, what parts of your other documents (e.g.
  requirements, hazard analysis, etc), it any, needed to be changed, and why?
  \item What are the limitations of your solution?  Put another way, given
  unlimited resources, what could you do to make the project better? (LO\_ProbSolutions)
  \item Give a brief overview of other design solutions you considered.  What
  are the benefits and tradeoffs of those other designs compared with the chosen
  design?  From all the potential options, why did you select the documented design?
  (LO\_Explores)
\end{enumerate}


\end{document}
