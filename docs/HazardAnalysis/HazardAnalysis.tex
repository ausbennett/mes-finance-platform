\documentclass{article}

\usepackage{booktabs}
\usepackage{pdflscape}
\usepackage{tabularx}
\usepackage{hyperref}
\usepackage{array}



\hypersetup{
    colorlinks=true,       % false: boxed links; true: colored links
    linkcolor=red,          % color of internal links (change box color with linkbordercolor)
    citecolor=green,        % color of links to bibliography
    filecolor=magenta,      % color of file links
    urlcolor=cyan           % color of external links
}

\title{Hazard Analysis\\\progname}

\author{\authname}

\date{}

%% Comments

\usepackage{color}

\newif\ifcomments\commentstrue %displays comments
%\newif\ifcomments\commentsfalse %so that comments do not display

\ifcomments
\newcommand{\authornote}[3]{\textcolor{#1}{[#3 ---#2]}}
\newcommand{\todo}[1]{\textcolor{red}{[TODO: #1]}}
\else
\newcommand{\authornote}[3]{}
\newcommand{\todo}[1]{}
\fi

\newcommand{\wss}[1]{\authornote{blue}{SS}{#1}} 
\newcommand{\plt}[1]{\authornote{magenta}{TPLT}{#1}} %For explanation of the template
\newcommand{\an}[1]{\authornote{cyan}{Author}{#1}}

%% Common Parts

\newcommand{\progname}{McMaster Engineering Society Custom Financial
Expense Reporting Platform} % PUT YOUR PROGRAM NAME HERE
\newcommand{\authname}{Team \#12, Reimbursement Rangers
\\ Adam Podolak
\\ Evan Sturmey
\\ Christian Petricca
\\ Austin Bennett
\\ Jacob Kish} % AUTHOR NAMES                  

\usepackage{hyperref}
    \hypersetup{colorlinks=true, linkcolor=blue, citecolor=blue, filecolor=blue,
                urlcolor=blue, unicode=false}
    \urlstyle{same}
                                


\begin{document}

\maketitle
\thispagestyle{empty}

~\newpage

\pagenumbering{roman}

\begin{table}[hp]
\caption{Revision History} \label{TblRevisionHistory}
\begin{tabularx}{\textwidth}{llX}
\toprule
\textbf{Date} & \textbf{Developer(s)} & \textbf{Change}\\
\midrule
Date1 & Name(s) & Description of changes\\
Date2 & Name(s) & Description of changes\\
... & ... & ...\\
\bottomrule
\end{tabularx}
\end{table}

~\newpage

\tableofcontents

~\newpage

\pagenumbering{arabic}

\wss{You are free to modify this template.}

\section{Introduction}

\wss{You can include your definition of what a hazard is here.}

\section{Scope and Purpose of Hazard Analysis}

\wss{You should say what \textbf{loss} could be incurred because of the
hazards.}

\section{System Boundaries and Components}

\wss{Dividing the system into components will help you brainstorm the hazards.
You shouldn't do a full design of the components, just get a feel for the major
ones.  For projects that involve hardware, the components will typically include
each individual piece of hardware.  If your software will have a database, or an
important library, these are also potential components.}

\section{Critical Assumptions}

\wss{These assumptions that are made about the software or system.  You should
minimize the number of assumptions that remove potential hazards.  For instance,
you could assume a part will never fail, but it is generally better to include
this potential failure mode.}


\section{Failure Mode and Effect Analysis}

\wss{Include your FMEA table here. This is the most important part of this document.}
\wss{The safety requirements in the table do not have to have the prefix SR.
The most important thing is to show traceability to your SRS. You might trace to
requirements you have already written, or you might need to add new
requirements.}
\wss{If no safety requirement can be devised, other mitigation strategies can be
entered in the table, including strategies involving providing additional
documentation, and/or test cases.}

\begin{landscape}

\begin{table}
\centering
\caption{FMEA for User Authentication and Access Control Design Function}
\begin{tabular}{|p{2.5cm}|p{3cm}|p{3cm}|p{5cm}|p{5cm}|p{2cm}|}
\hline
\textbf{Design function} & \textbf{Failure Modes} & \textbf{Effects of Failure} & \textbf{Causes of Failure} & \textbf{Recommended Actions} & \textbf{Req.} \\ \hline

%Row 1
\parbox[t]{2.5cm}{\raggedright User Authentication and Access Control} & \parbox[t]{3cm}{\raggedright Unauthorized user access} & \parbox[t]{3cm}{\raggedright Unauthorized users gain access to sensitive data} &
\parbox[t]{5cm}{\raggedright
    \begin{enumerate}
      \item[a.] Weak password policies/restrictions
      \item[b.] Lack of multi-factor authentication
    \end{enumerate}
  } &
\parbox[t]{5cm}{\raggedright
    \begin{enumerate}
        \item[a.] Enforce users to create strong passwords
        \item[b.] Implement and enforce multi-factor authentication
    \end{enumerate}
} &

\parbox[t]{2cm}{\raggedright
    \begin{enumerate}
        \item[a.] SCR2
        \item[b.] PRD2
    \end{enumerate}
}
\\ \hline

%Row 2
&
\parbox[t]{3cm}{\raggedright User accounts are compromised} &
\parbox[t]{3cm}{\raggedright Data theft, manipulation, or system sabotage} &
\parbox[t]{5cm}{\raggedright
    \begin{enumerate}
      \item[a.] Weak or reused passwords
    \end{enumerate}
  } &
\parbox[t]{5cm}{\raggedright
    \begin{enumerate}
      \item[a.] Enforce strong password policies
    \end{enumerate}
  } &
\parbox[t]{2cm}{\raggedright
    \begin{enumerate}
        \item[a.] SCR1
    \end{enumerate}
} \\ \hline

%Row 3
&
\parbox[t]{3cm}{\raggedright Improper privilege escalation by users} &
\parbox[t]{3cm}{\raggedright Users perform actions beyond their authorization} &
\parbox[t]{5cm}{\raggedright
    \begin{enumerate}
      \item[a.] Missing or incorrect configuration of access controls
    \end{enumerate}
  } &
\parbox[t]{5cm}{\raggedright
    \begin{enumerate}
      \item[a.] Use role-based access control
    \end{enumerate}
  } &
\parbox[t]{2cm}{\raggedright
    \begin{enumerate}
        \item[a.] AR1
    \end{enumerate}
} \\ \hline

\end{tabular}
\end{table}


\begin{table}
\centering
\caption{FMEA for Data Privacy and Compliance Design Function}
\begin{tabular}{|p{2.5cm}|p{3cm}|p{3cm}|p{5cm}|p{5cm}|p{2cm}|}
\hline
\textbf{Design function} & \textbf{Failure Modes} & \textbf{Effects of Failure} & \textbf{Causes of Failure} & \textbf{Recommended Actions} & \textbf{Req.} \\ \hline

%Row 1
\parbox[t]{2.5cm}{\raggedright Data Privacy and Compliance} & \parbox[t]{3cm}{\raggedright Exposure of confidential personal financial information} & \parbox[t]{3cm}{\raggedright Legal/financial penalties and damage to the organization's reputation} &
\parbox[t]{5cm}{\raggedright
    \begin{enumerate}
      \item[a.] Poor security practices
      \item[b.] Lack of encryption of data in transit 
    \end{enumerate}
  } &
\parbox[t]{5cm}{\raggedright
    \begin{enumerate}
        \item[a.] Encrypt all data containing sensitive financial information at rest and in transit
    \end{enumerate}
} &

\parbox[t]{2cm}{\raggedright
    \begin{enumerate}
        \item[a.] PVR1
    \end{enumerate}
}
\\ \hline

%Row 2
&
\parbox[t]{3cm}{\raggedright Not complying with regulations relating to data protection} &
\parbox[t]{3cm}{\raggedright Repercussions from regulatory bodies, potential shutdown of organization's services} &
\parbox[t]{5cm}{\raggedright
    \begin{enumerate}
      \item[a.] Lack of compliance measures
      \item[b.] Not adhering to legal requirements
    \end{enumerate}
  } &
\parbox[t]{5cm}{\raggedright
    \begin{enumerate}
      \item[a.] Regular reviews and updates for policies regarding data compliance
    \end{enumerate}
  } &
\parbox[t]{2cm}{\raggedright
    \begin{enumerate}
        \item[a.] LR1
        \item[b.] LR2
    \end{enumerate}
} \\ \hline

\end{tabular}
\end{table}



\begin{table}
\centering
\caption{FMEA for Data Backup and Recovery Design Function}
\begin{tabular}{|p{2.5cm}|p{3cm}|p{3cm}|p{5cm}|p{5cm}|p{2cm}|}
\hline
\textbf{Design function} & \textbf{Failure Modes} & \textbf{Effects of Failure} & \textbf{Causes of Failure} & \textbf{Recommended Actions} & \textbf{Req.} \\ \hline

%Row 1
\parbox[t]{2.5cm}{\raggedright Data Backup and Recovery} & \parbox[t]{3cm}{\raggedright Scheduled data backup fails to execute or complete} & \parbox[t]{3cm}{\raggedright Financial data is lost} &
\parbox[t]{5cm}{\raggedright
    \begin{enumerate}
      \item[a.] Errors in the configuration of backup schedule automation
      \item[b.] General software errors or bugs
    \end{enumerate}
  } &
\parbox[t]{5cm}{\raggedright
    \begin{enumerate}
        \item[a.] Regularly monitor backup processes
        \item[b.] Conduct regular integration testing throughout development
    \end{enumerate}
} &

\parbox[t]{2cm}{\raggedright
    \begin{enumerate}
        \item[a.] RFT1
        \item[b.] MR1
    \end{enumerate}
}
\\ \hline

%Row 2
&
\parbox[t]{3cm}{\raggedright Data that is backed up is corrupted} &
\parbox[t]{3cm}{\raggedright Inability to retrieve data or restore system functionality promptly} &
\parbox[t]{5cm}{\raggedright
    \begin{enumerate}
      \item[a.] Remote database failures
    \end{enumerate}
  } &
\parbox[t]{5cm}{\raggedright
    \begin{enumerate}
      \item[a.] Store backed-up data in multiple, secure locations
    \end{enumerate}
  } &
\parbox[t]{2cm}{\raggedright
    \begin{enumerate}
        \item[a.] IMM1
    \end{enumerate}
} \\ \hline

\end{tabular}
\end{table}


\end{landscape}





\section{Safety and Security Requirements}

\wss{Newly discovered requirements.  These should also be added to the SRS.  (A
rationale design process how and why to fake it.)}

\section{Roadmap}

\wss{Which safety requirements will be implemented as part of the capstone timeline?
Which requirements will be implemented in the future?}

\newpage{}

\section*{Appendix --- Reflection}

\wss{Not required for CAS 741}

The purpose of reflection questions is to give you a chance to assess your own
learning and that of your group as a whole, and to find ways to improve in the
future. Reflection is an important part of the learning process.  Reflection is
also an essential component of a successful software development process.  

Reflections are most interesting and useful when they're honest, even if the
stories they tell are imperfect. You will be marked based on your depth of
thought and analysis, and not based on the content of the reflections
themselves. Thus, for full marks we encourage you to answer openly and honestly
and to avoid simply writing ``what you think the evaluator wants to hear.''

Please answer the following questions.  Some questions can be answered on the
team level, but where appropriate, each team member should write their own
response:


\begin{enumerate}
    \item What went well while writing this deliverable? 

    \textbf{Adam:} For my section (FMEA tables) brainstorming the actual hazards/risks for the system went relatively well. There aren't any inherent "safety" risks that would endanger anyone's physical well-being, there really are only security risks that we have to worry about. Since we're dealing with financial information, these risks were quite obvious to identify and mitigate. Evidently, we need encryption, user authentication, etc. to ensure that we're upholding data privacy and financial information is stored securely.
    
    \item What pain points did you experience during this deliverable, and how
    did you resolve them?

    \textbf{Adam:} One of the biggest pain points I had while writing my section was formatting the tables using latex. I had to import extra packages and spend a lot of time playing around with the spacing of things to ensure it was legible and formatted decently. Unfortunately, This took more time than the actual engineering work of coming up with the hazards. It would be nice if we could make the tables in a Word document and then insert a screenshot into the Latex file, but I can see how this would defeat the purpose of traceability.
    
    \item Which of your listed risks had your team thought of before this
    deliverable, and which did you think of while doing this deliverable? For
    the latter ones (ones you thought of while doing the Hazard Analysis), how
    did they come about?
    \item Other than the risk of physical harm (some projects may not have any
    appreciable risks of this form), list at least 2 other types of risk in
    software products. Why are they important to consider?
\end{enumerate}

\end{document}