\documentclass[12pt, titlepage]{article}
%
\usepackage{booktabs}
\usepackage{tabularx}
\usepackage{hyperref}
\usepackage{float}
\usepackage{soul}
\hypersetup{
    colorlinks,
    citecolor=black,
    filecolor=black,
    linkcolor=red,
    urlcolor=blue
}
\usepackage[round]{natbib}

%% Comments

\usepackage{color}

\newif\ifcomments\commentstrue %displays comments
%\newif\ifcomments\commentsfalse %so that comments do not display

\ifcomments
\newcommand{\authornote}[3]{\textcolor{#1}{[#3 ---#2]}}
\newcommand{\todo}[1]{\textcolor{red}{[TODO: #1]}}
\else
\newcommand{\authornote}[3]{}
\newcommand{\todo}[1]{}
\fi

\newcommand{\wss}[1]{\authornote{blue}{SS}{#1}} 
\newcommand{\plt}[1]{\authornote{magenta}{TPLT}{#1}} %For explanation of the template
\newcommand{\an}[1]{\authornote{cyan}{Author}{#1}}

%% Common Parts

\newcommand{\progname}{McMaster Engineering Society Custom Financial
Expense Reporting Platform} % PUT YOUR PROGRAM NAME HERE
\newcommand{\authname}{Team \#12, Reimbursement Rangers
\\ Adam Podolak
\\ Evan Sturmey
\\ Christian Petricca
\\ Austin Bennett
\\ Jacob Kish} % AUTHOR NAMES                  

\usepackage{hyperref}
    \hypersetup{colorlinks=true, linkcolor=blue, citecolor=blue, filecolor=blue,
                urlcolor=blue, unicode=false}
    \urlstyle{same}
                                


\begin{document}

\title{Verification and Validation Report: \progname} 
\author{\authname}
\date{\today}
	
\maketitle

\pagenumbering{roman}

\section{Revision History}

\begin{tabularx}{\textwidth}{p{3cm}p{2cm}X}
\toprule {\bf Date} & {\bf Version} & {\bf Notes}\\
\midrule
3/11/2025 & 1.0 & Initial revision, including all sections and reflection\\
3/24/2025 & 2.0 & Updating for Authentication Module, see commit: \href{https://github.com/ausbennett/mes-finance-platform/commit/4cef88de4d7c2a664fa8e2395804251c0a6baeb4}{4cef88d}\\
\bottomrule
\end{tabularx}

~\newpage

% \section{Symbols, Abbreviations and Acronyms}

% \renewcommand{\arraystretch}{1.2}
% \begin{tabular}{l l} 
%   \toprule		
%   \textbf{symbol} & \textbf{description}\\
%   \midrule 
%   T & Test\\
%   \bottomrule
% \end{tabular}\\

\wss{symbols, abbreviations or acronyms -- you can reference the SRS tables if needed}

\newpage

\tableofcontents

\listoftables %if appropriate

\newpage

\pagenumbering{arabic}

\newpage

\section{Functional Requirements Evaluation}

The functional requirements can be referenced from the \href{https://github.com/ausbennett/mes-finance-platform/blob/main/docs/SRS/SRS.pdf}{SRS}. The modules can be referenced below and from the \href{https://github.com/ausbennett/mes-finance-platform/blob/main/docs/Design/SoftArchitecture/MG.pdf}{Module Guide}:

\begin{itemize}
    \item M1: Hardware-Hiding Module
    \item M2: Account Management Module
    \item M3: Requests Module
    \item M4: Notification Module
    \item M5: User Dashboard Module
    \item M6: \st{Authentication Module}
    \item M7: Email Module
    \item M8: Account Management Controller Module
    \item M9: Requests Controller Module
    \item M10: Clubs Database
    \item M11: Users Database
    \item M12: Requests Database
    \item M13: Graphical User Interface
\end{itemize}

Table~\ref{tab:FReval} summarizes the functional requirements checklist for each functional requirement from the SRS, linking to each module from the MG and test case from the \href{https://github.com/ausbennett/mes-finance-platform/blob/main/docs/VnVPlan/VnVPlan.pdf}{VnV Plan}. If a functional requirement was determined to be satisfied or not staisfied by the tests with a "*" flag, then more explanation is given below:

\begin{itemize}
    \item FR1.3: UI components and pages are implemented to allow users and admins to edit submitted requests, however, work must now be done to connect backend to frontend UI to allow for changes to be reflected in the database.
    \item FR1.5: The system does not currently notify users of changes to the status of their reimbursement request, however, the email notification service/module is set up and can be executed in an isolated environment. Work must now be done ensure the email service can dynamically email users with updates for their submitted requests.
    \item FR1.6: UI is implemented to allow users to select a budget category. Budget categories need to be dynamically pulled from a database collection. Work must be done to connect backend service (Plaid API) to frontend UI to allow budget categories to be linked to requests and reflected in the database.
    \item \st{FR1.7: Due to the fact that authentication is not completely implemented, this requirement cannot be satisfied. We do have page routing and UI implemented to distinguish admin page views and user page views.} \textit{Note: it has been decided we will integrate into the existing MES authentication, which shall satisfy this requirement.}
    \item \st{FR3.2: Authentication is not fully set up. Currently, we are manually setting user browser tokens when creating a user and logging in, work must be done to either dynamically assign user browser token, or adjust the authentication process to user vanilla username (email) and password for the login process.} \textit{See note above.}
\end{itemize}

\begin{table}[ht]
    \centering
    \footnotesize
    \caption{Functional Requirements Evaluation}
    \label{tab:FReval}
    \begin{tabular}{|p{1.2cm}|p{2.7cm}|p{2.4cm}|p{2.2cm}|p{1.4cm}|}
    \hline
    \textbf{FR} & \textbf{Modules} & \textbf{Test Suite} & \textbf{Test Cases} & \textbf{Satisfied (Y/N)} \\
    \hline
    FR1.1 & M3, M5, M9, M10, M11, M12, M13 & reconciler.test.js & FR1.1-TC1, FR1.1-TC2 & Yes \\
    \hline
    FR1.2 & M3, M5, M9, M10, M11, M12, M13 & requests.test.js & FR1.2-TC1, FR1.2-TC2 & Yes\\
    \hline
    FR1.3 & M3, M5, M9, M10, M11, M12, M13 & requests.test.js & FR1.3-TC1 & Yes*\\
    \hline
    FR1.4 & M3, M5, M9, M10, M11, M12, M13 & reconciler.test.js & FR1.4-TC1 & Yes\\
    \hline
    FR1.5 & M4, M7 & emailer.test.js & FR1.5-TC1 & No*\\
    \hline
    FR1.6 & M12 & reconciler.test.js & FR1.6-TC1 & No*\\
    \hline
    FR1.7 & M6 & Acct. mgmt. TS (TBD) & FR1.7-TC1, FR1.7-TC2 & No*\\
    \hline
    FR1.8 & M5, M3, M9, M13 & requests.test.js & FR1.8-TC1 & No\\
    \hline
    FR1.9 & M12 & reconciler.test.js & FR1.9-TC1 & No\\
    \hline
    FR3.1 & M10, M11 & Acct. mgmt. TS (TBD) & FR3.1-TC1 & Yes\\
    \hline
    FR3.2 & M6 & emailer.test.js, Acct. mgmt. TS (TBD) & FR3.2-TC1, FR3.2-TC2 & No*\\
    \hline
    FR3.3 & M2, M8 & & FR3.3-TC1 & Yes\\
    \hline
    \end{tabular}
    \normalsize
\end{table}

\newpage

\section{Nonfunctional Requirements Evaluation}

\subsection{Look and Feel (NFR10)}  
\label{nfr10}

\begin{itemize}
    \item \textbf{Appearance (Test T1):}  
    \begin{itemize}
        \item \textbf{Checklist 6.1.1.1:} UI renders properly on desktops (1024x768) – \textbf{Passed}.  
        \item \textbf{Checklist 6.1.1.2:} UI renders properly on mobile devices – \textbf{Failed}.  
        \item \textbf{Checklist 6.1.1.3:} All elements aligned on all devices – \textbf{Failed}.  
        \item \textbf{Checklist 6.1.1.4:} Fonts, colors, spacing consistent – \textbf{Passed}.  
        \item \textbf{Checklist 6.1.1.5:} Branding aligns with MES guidelines – \textbf{Passed}.  
        \item \textbf{Checklist 6.1.1.8:} Key information displayed prominently – \textbf{Passed}.  
    \end{itemize}
    
    \item \textbf{Style (Test T1):}  
    \begin{itemize}
        \item \textbf{Checklist 6.1.1.6:} Icons intuitive – \textbf{Passed}.  
        \item \textbf{Checklist 6.1.1.7:} Tooltips for unclear icons – \textbf{Passed}.  
    \end{itemize}
\end{itemize}

\subsubsection*{Notes}  
\begin{itemize}
    \item \textbf{Failures in Appearance T1:}  
    \begin{itemize}
        \item Mobile rendering exhibited layout inconsistencies on screens smaller than 5.5 inches (Checklist 6.1.1.2).  
        \item Elements misaligned on mobile devices during form submissions (Checklist 6.1.1.3).  
    \end{itemize}
    \item \textbf{Successes in Style T1:} Users recognized all icons intuitively, and tooltips were provided for ambiguous symbols (Checklist 6.1.1.6–7).  
\end{itemize}

\subsection{Usability and Humanity (NFR11)}  
\label{nfr11}

\begin{itemize}
    \item \textbf{Ease of Use (Tests T1, T2):}  
    \begin{itemize}
        \item \textbf{Checklist 6.1.2.1:} Navigate to primary functions in $\le$ 3 clicks – \textbf{Passed}.  
        \item \textbf{Checklist 6.1.2.2:} Form validation messages correct – \textbf{Passed}.  
        \item \textbf{Checklist 6.1.2.3:} Complete tasks in $\le$ 2 minutes – \textbf{Passed}.  
        \item \textbf{Checklist 6.1.2.4:} Edit profile in $\le$ 3 clicks – \textbf{Passed}.  
        \item \textbf{Checklist 6.1.2.5:} Personalization changes reflected immediately – \textbf{Passed}.  
        \item \textbf{Checklist 6.1.2.6:} Tutorial video accessible – \textbf{Failed}.  
        \item \textbf{Checklist 6.1.2.7:} Instructions free of jargon – \textbf{Passed}.  
        \item \textbf{Checklist 6.1.2.8:} Button labels clear – \textbf{Passed}.  
    \end{itemize}
    
    \item \textbf{Accessibility (Test T1, T2):}  
    \begin{itemize}
        \item \textbf{Checklist 6.1.2.9:} Keyboard navigation equivalent to mouse – \textbf{Passed}.  
        \item \textbf{Checklist 6.1.2.10:} Text font size $\ge$ 12pt – \textbf{Passed}.  
    \end{itemize}
\end{itemize}

\subsubsection*{Notes}  
\begin{itemize}
    \item \textbf{Failure in Ease of Use (Checklist 6.1.2.6):} The tutorial video was not easily accessible from the home page.  
    \item \textbf{Testing Context:} Results for NFR10 and NFR11 were conducted within the group. Further testing will involve the supervisor and potential users.  
\end{itemize}

\subsection{Performance (NFR12)}  
\label{nfr12}

\begin{itemize}
    \item \textbf{Speed and Latency (Tests T1, T2, T3):}  
    \begin{itemize}
        \item \textbf{Checklist 6.1.3.1:} 95\% of tasks completed in < 1s – \textbf{Passed}.  
        \item \textbf{Checklist 6.1.3.2:} Large files processed in $\le$ 30s – \textbf{Failed}.  
        \item \textbf{Checklist 6.1.3.3:} System recovers from crash without data loss – \textbf{Passed}.  
        \item \textbf{Checklist 6.1.3.4:} Backups stored and verified – \textbf{Failed}.  
        \item \textbf{Checklist 6.1.3.5:} Handles 20 simultaneous users – \textbf{Failed}.  
        \item \textbf{Checklist 6.1.3.6:} Processes 10 years of dummy data – \textbf{Failed}.  
    \end{itemize}
    
    \item \textbf{Safety-Critical (Test T1, T2):}  
    \begin{itemize}
        \item \textbf{Checklist 6.1.3.7:} Credit card numbers masked – \textbf{Failed}.  
        \item \textbf{Checklist 6.1.3.8:} Unauthorized access prevented – \textbf{Failed}.  
    \end{itemize}
    
    \item \textbf{Precision and Accuracy (Test T1):}  
    \begin{itemize}
        \item \textbf{Checklist 6.1.3.9:} Monetary calculations rounded to 2 decimal places – \textbf{Passed}.  
    \end{itemize}
    
    \item \textbf{Robustness or Fault-Tolerance (Test T1):}  
    \begin{itemize}
        \item \textbf{Checklist 6.1.3.10:} Daily backups verified – \textbf{Failed}.  
    \end{itemize}
    
    \item \textbf{Capacity (Tests T1, T2, T3):}  
    \begin{itemize}
        \item \textbf{Checklist 6.1.3.11:} Handles large request volume – \textbf{Failed}.  
        \item \textbf{Checklist 6.1.3.12:} Handles simultaneous requests – \textbf{Failed}.  
        \item \textbf{Checklist 6.1.3.13:} Stores 10 years of dummy data – \textbf{Failed}.  
    \end{itemize}
    
    \item \textbf{Scalability or Extensibility (Tests T1, T2):}  
    \begin{itemize}
        \item \textbf{Checklist 6.1.3.14:} Handles stress tests until failure – \textbf{Failed}.  
        \item \textbf{Checklist 6.1.3.15:} Third-party endpoints return JSON – \textbf{Passed}.  
    \end{itemize}
    
    \item \textbf{Longevity (Test T1):}  
    \begin{itemize}
        \item \textbf{Checklist 6.1.3.16:} Code components adaptable for future maintenance – \textbf{Passed}.  
    \end{itemize}
\end{itemize}

\subsubsection*{Notes}  
\begin{itemize}
    \item \textbf{Failures in Development Environment:} Many failed test cases (e.g., capacity, simultaneous users, backups) are attributed to the system being in development rather than production.
\end{itemize}

\subsection{Operational and Environmental (NFR13)}  
\label{nfr13}

\begin{itemize}
    \item \textbf{Expected Physical Environment (Test T1):}  
    \begin{itemize}
        \item \textbf{Checklist 6.1.4.1:} Compatibility with Windows 10 and Intuit QuickBooks specs – \textbf{Deferred}.  
    \end{itemize}
    
    \item \textbf{Wider Environment (Test T1):}  
    \begin{itemize}
        \item \textbf{Checklist 6.1.4.2:} Compliance with Canadian financial institution standards – \textbf{Deferred}.  
    \end{itemize}
    
    \item \textbf{Interfacing with Adjacent Systems (Test T1):}  
    \begin{itemize}
        \item \textbf{Checklist 6.1.4.3:} Integration with major Canadian banks – \textbf{Deferred}.  
    \end{itemize}
    
    \item \textbf{Production (Test T1):}  
    \begin{itemize}
        \item \textbf{Checklist 6.1.4.4:} User training material clarity – \textbf{Deferred}.  
    \end{itemize}
    
    \item \textbf{Release (Test T1):}  
    \begin{itemize}
        \item \textbf{Checklist 6.1.4.5:} Feasibility of biannual releases – \textbf{Deferred}.  
    \end{itemize}
\end{itemize}

\subsubsection*{Notes}  
\begin{itemize}
    \item \textbf{Deferred Testing:} Tests for NFR13 will align with the system’s progression to production readiness. Current development constraints (e.g., incomplete integrations, non-finalized environments) prevent accurate assessment.  
    \item \textbf{Next Steps:} These tests will be prioritized during later development phases, including post-deployment monitoring and stakeholder reviews.  
\end{itemize}

\subsection{Maintainability and Support (NFR14)}  
\label{nfr14}

\begin{itemize}
    \item \textbf{Maintenance (Test T1):}  
    \begin{itemize}
        \item \textbf{Checklist 6.1.4.1:} Coding standards followed – \textbf{Passed}.  
        \item \textbf{Checklist 6.1.4.2:} No critical errors in build tools – \textbf{Passed}.  
        \item \textbf{Checklist 6.1.4.3:} Documentation up to date – \textbf{Passed}.  
        \item \textbf{Checklist 6.1.4.4:} Updates tested and documented – \textbf{Passed}.  
    \end{itemize}
    
    \item \textbf{Supportability (Test T1):}  
    \begin{itemize}
        \item \textbf{Checklist 6.1.4.5:} Documentation updated with releases – \textbf{Passed}.  
    \end{itemize}
    
    \item \textbf{Adaptability (Test T1):}  
    \begin{itemize}
        \item \textbf{Checklist 6.1.4.6:} Regression tests performed regularly – \textbf{Failed}.  
    \end{itemize}
\end{itemize}

\subsection{Security (NFR15)}  
\label{nfr15}

\begin{itemize}
    \item \textbf{Access (Tests T1, T2):}  
    \begin{itemize}
        \item \textbf{Checklist 6.1.5.1:} Unauthorized access prevented – \textbf{Passed}.  
        \item \textbf{Checklist 6.1.5.2:} Multi-factor authentication enforced – \textbf{Failed}.  
    \end{itemize}
    
    \item \textbf{Integrity/Audit (Test T1):}  
    \begin{itemize}
        \item \textbf{Checklist 6.1.5.3:} Audit logs maintained – \textbf{Failed}.  
    \end{itemize}
    
    \item \textbf{Privacy (Test T1):}  
    \begin{itemize}
        \item \textbf{Checklist 6.1.5.4:} Sensitive data encrypted – \textbf{Failed}.  
    \end{itemize}
    
    \item \textbf{Immunity (Test T1):}  
    \begin{itemize}
        \item \textbf{Checklist 6.1.5.5:} System recovers from attacks within 4 hours – \textbf{Failed}.  
    \end{itemize}
\end{itemize}

\subsection{Cultural (NFR16)}  
\label{nfr16}

\begin{itemize}
    \item \textbf{Testing Status:} Cultural compliance testing (e.g., internationalization, accessibility for non-English speakers) will be conducted during future development phases.  
\end{itemize}

\subsection{Compliance (NFR17)}  
\label{nfr17}

\begin{itemize}
    \item \textbf{Legal (Test T1):}  
    \begin{itemize}
        \item \textbf{Checklist 6.1.6.1:} Adheres to data protection policies – \textbf{Failed}.  
        \item \textbf{Checklist 6.1.6.2:} Compliant with Canadian financial regulations – \textbf{Failed}.  
    \end{itemize}
    
    \item \textbf{Standard Compliance (Test T1):}  
    \begin{itemize}
        \item \textbf{Checklist 6.1.6.3:} Follows web security standards – \textbf{Failed}.  
    \end{itemize}
\end{itemize}

\subsubsection*{Notes}  
\begin{itemize}
    \item \textbf{Deferred Fixes:} Compliance failures (data protection, financial regulations, web security) will be prioritized during Rev 1 development.  
\end{itemize}

\section{Unit Testing}

Most files, excluding the 3rd-party library, top-level and GUI modules had corresponding unit tests.
All tests were performed using jest for testing javascript, and each commit to the main branch must pass all the tests.
All the tests passed as seen in the Test Report \ref{test_report}.

\section{Changes Due to Testing}

\wss{This section should highlight how feedback from the users and from 
the supervisor (when one exists) shaped the final product.  In particular 
the feedback from the Rev 0 demo to the supervisor (or to potential users) 
should be highlighted.}

During Rev0 presentations, as well as meetings with the our supervisor, it is made apparent that authentication will be an "assumed" feature and not necessary to be show cased during final demos or expo. With that being said, we've decided as a team to put any further authentication changes on the backlog. As MES currently has a express.js authentication system that may likely seamlessly integrate into our current project. An upcoming change that will be implemented, is caching of the plaid API, this was feedback from our supervisor when showcasing the plaid API integration. 
We also conducted usability testing with a fellow engineering student. The documentation for this testing can be found at:
\url{https://github.com/ausbennett/mes-finance-platform/tree/main/docs/Extras/UsabilityTest}. Three key pieces of feedback were presented after. Firstly, the student found clicking the MES logo button to return to the landing page was unintuitive and initially struggled with this navigation. In response, we added a new "Dashboard" button in the top right to explicitly tell the user how to return (note that the MES button is still present and navigating to the dashboard). Secondly, the student expressed that our red buttons on the new request page were too bright and did not contrast well with the white page background. We updated the buttons to a be darker maroon with white lettering, which provides much better contrast and conforms to our McMaster colours. Finally, the user did not like our edit account button, which was simply the word "Account". We agreed that it was unappealing and changed it to an icon of a person, which is intuitive and removed unnecessary text.

During our final presentation, we also received feedback. It was pointed out that the "total amount" field in our new request forms is entered manually, instead of being autofilled by adding up the individual amounts. This creates opportunities for user error in mistakenly adding or approving incorrect amounts. Furthermore, it was expressed that on our audit page, our click-and-drag reconciliation system may be difficult to use if there are a large number of entries to scroll through. In response to these comments, we intend to autofill the total amount field to prevent mistakes. We also would like to add an option for presenting the audit page in different views, and adding a "Compact" or "List" view to make reconciling large amounts of data easier. However, these changes will have to be implemented in the future as we have already given our final presentation and now focused on cleaning our documentation and preparing for the expo.

\section{Automated Testing}

The tests were set up to automatically run in GitHub Actions whenever a commit was pushed to the $\mathtt{main}$ branch.
The configuration for the CI/CD automation can be found at \url{https://github.com/ausbennett/mes-finance-platform/blob/main/.github/workflows/test.yml}
and the test pipeline can be viewed at \url{https://github.com/ausbennett/mes-finance-platform/actions/workflows/test.yml}.

\newpage

\section{Trace to Requirements}
\begin{table}[!h]
\begin{table}[H]
\begin{tabularx}{\textwidth}{p{5cm}p{5cm}}
\toprule {\bf Test Suite} & {\bf Functional Requirement}\\
  emailer.test.js & FR1.5, FR3.2\\
  reconciler.test.js & FR1.1, FR1.4, FR1.6, FR1.9, INR1, SPLR1-3\\
  requests.test.js & FR1.2, FR1.3, FR1.8, PVR1\\ 
  Account management test suite (TBD) & FR1.7, FR3.1, FR3.2\\
  Usability testing (via checklist) & APR1, APR2, STYR1, STYR2, EUR1-3, PIR1, LER2, UAPR1, ACSR1, SCR1-5, CLTR1\\
  Static analysis (via checklist) & POAR1, LOR1, WR1, RLR1, ACSR2, MR2, SR1, LR1, LR2, STR1\\
  Load testing & CPR1-3, SER2\\
\end{tabularx}
\caption{Mapping between test suites and requirements}
\end{table}
\end{table}

\newpage

\section{Trace to Modules}
\begin{table}[H]
\begin{tabularx}{\textwidth}{p{5cm}p{5cm}}
\toprule {\bf Test Suite} & {\bf Module}\\
\midrule 
emailer.test.js & Notification Module (M4), Authentication Module (M6), Emailer API (M7)\\
reconciler.test.js & Clubs, User, Request Database(M10-12)\\
requests.test.js & Requests Module(M3), Requests Controller (M9)\\
Account management test suite (TBD) & Account Management Module (M2), User Dashboard Module (M5), Account Management Controller (M8)\\
Usability testing (via checklist) & Graphical User Interface (M13)\\
\bottomrule
\end{tabularx}
\caption{Mapping between test suites and modules}
\end{table}

\newpage

\section{Code Coverage Metrics}

Code coverage is shown in the Test Report in section \ref{test_report}.

\section{Appendix}
\subsection{Test Report} \label{test_report}
\begin{small} 
  \begin{verbatim} 
     PASS  tests/requests.test.js
      Request Endpoints and Model Validations
        Model Validations
          should require necessary reimbursement fields (3 ms)
        Reimbursement Endpoints
          should create reimbursement with file upload (34 ms)

    Test Suites: 1 passed, 1 total
    Tests:       2 passed, 2 total
    Snapshots:   0 total
    Time:        0.927 s, estimated 1 s
    Ran all test suites.
  \end{verbatim}
\end{small}

\newpage{}
\section*{Appendix --- Reflection}

The information in this section will be used to evaluate the team members on the
graduate attribute of Reflection.

The purpose of reflection questions is to give you a chance to assess your own
learning and that of your group as a whole, and to find ways to improve in the
future. Reflection is an important part of the learning process.  Reflection is
also an essential component of a successful software development process.  

Reflections are most interesting and useful when they're honest, even if the
stories they tell are imperfect. You will be marked based on your depth of
thought and analysis, and not based on the content of the reflections
themselves. Thus, for full marks we encourage you to answer openly and honestly
and to avoid simply writing ``what you think the evaluator wants to hear.''

Please answer the following questions.  Some questions can be answered on the
team level, but where appropriate, each team member should write their own
response:


\begin{enumerate}
  \item What went well while writing this deliverable?\\
\\
In this deliverable, leveraging tools like GitHub Actions for automated testing helped us streamline validation. Additionally, having already incorporated feedback from the V\&V Plan, it made the process of creating the V\&V Report much easier. From delegating roles to having pre-made checklists for usability testing, using the V\&V Plan made our lives much easier.\\

  \item What pain points did you experience during this deliverable, and how
    did you resolve them?\\
\\
A significant challenge was designing a test suite that remains relevant as the project evolves. Since some core features are still in development, early tests risked becoming obsolete as APIs and interfaces changed. To address this, we prioritized testing isolated modules (e.g., database schemas, utility functions) with stable contracts and used mocking for incomplete subsystems. We also adopted a CI/CD-driven iterative approach, updating tests incrementally alongside implementation sprints to ensure alignment while preserving test value.\\

  \item Which parts of this document stemmed from speaking to your client(s) or
  a proxy (e.g. your peers)? Which ones were not, and why?\\
\\
The usability testing checklist was shaped by stakeholders as our supervisor has been vocal about which features should have a bigger focus on usability rather than expanding the functionality. Other things, such as our automated testing strategy, were internally designed. From the testing we did before the Rev0 demo, we are already aware of which areas require more rigorous testing compared to other features.\\
  
  \item In what ways was the Verification and Validation (VnV) Plan different
  from the activities that were actually conducted for VnV?  If there were
  differences, what changes required the modification in the plan?  Why did
  these changes occur?  Would you be able to anticipate these changes in future
  projects?  If there weren't any differences, how was your team able to clearly
  predict a feasible amount of effort and the right tasks needed to build the
  evidence that demonstrates the required quality?  (It is expected that most
  teams will have had to deviate from their original VnV Plan.)\\
\\The main difference between the plan and the report was the scale of the testing. The plan was more ambitious than what was probably doable in the time frame, and we had pages and pages of tests for NFRs that would have required external stakeholders to execute, some of whom may be hard to get a hold of. As we moved forward it became clear we needed to focus our energy toward a smaller suite that makes more use of automated testing and usability testing with Luke to cover our bases. It is definitely possible in future projects to foresee this kind of an issue. However, timing can be a difficult thing to predict and one often doesn't know how things will pan out until they actually happen.
\end{enumerate}

\end{document}
