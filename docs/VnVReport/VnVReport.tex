\documentclass[12pt, titlepage]{article}

\usepackage{booktabs}
\usepackage{tabularx}
\usepackage{hyperref}
\hypersetup{
    colorlinks,
    citecolor=black,
    filecolor=black,
    linkcolor=red,
    urlcolor=blue
}
\usepackage[round]{natbib}

\input{../Comments}
%% Common Parts

\newcommand{\progname}{McMaster Engineering Society Custom Financial
Expense Reporting Platform} % PUT YOUR PROGRAM NAME HERE
\newcommand{\authname}{Team \#12, Reimbursement Rangers
\\ Adam Podolak
\\ Evan Sturmey
\\ Christian Petricca
\\ Austin Bennett
\\ Jacob Kish} % AUTHOR NAMES                  

\usepackage{hyperref}
    \hypersetup{colorlinks=true, linkcolor=blue, citecolor=blue, filecolor=blue,
                urlcolor=blue, unicode=false}
    \urlstyle{same}
                                


\begin{document}

\title{Verification and Validation Report: \progname} 
\author{\authname}
\date{\today}
	
\maketitle

\pagenumbering{roman}

\section{Revision History}

\begin{tabularx}{\textwidth}{p{3cm}p{2cm}X}
\toprule {\bf Date} & {\bf Version} & {\bf Notes}\\
\midrule
Date 1 & 1.0 & Notes\\
Date 2 & 1.1 & Notes\\
\bottomrule
\end{tabularx}

~\newpage

\section{Symbols, Abbreviations and Acronyms}

\renewcommand{\arraystretch}{1.2}
\begin{tabular}{l l} 
  \toprule		
  \textbf{symbol} & \textbf{description}\\
  \midrule 
  T & Test\\
  \bottomrule
\end{tabular}\\

\wss{symbols, abbreviations or acronyms -- you can reference the SRS tables if needed}

\newpage

\tableofcontents

\listoftables %if appropriate

\listoffigures %if appropriate

\newpage

\pagenumbering{arabic}

\newpage

\section{Functional Requirements Evaluation}

The functional requirements can be referenced from the \href{https://github.com/ausbennett/mes-finance-platform/blob/main/docs/SRS/SRS.pdf}{SRS}. The modules can be referenced below and from the \href{https://github.com/ausbennett/mes-finance-platform/blob/main/docs/Design/SoftArchitecture/MG.pdf}{Module Guide}:

\begin{itemize}
    \item M1: Hardware-Hiding Module
    \item M2: Account Management Module
    \item M3: Requests Module
    \item M4: Notification Module
    \item M5: User Dashboard Module
    \item M6: Authentication Module
    \item M7: Email Module
    \item M8: Account Management Controller Module
    \item M9: Requests Controller Module
    \item M10: Clubs Database
    \item M11: Users Database
    \item M12: Requests Database
    \item M13: Graphical User Interface
\end{itemize}

Table~\ref{tab:FReval} summarizes the functional requirements checklist for each functional requirement from the SRS, linking to each module from the MG and test case from the \href{https://github.com/ausbennett/mes-finance-platform/blob/main/docs/VnVPlan/VnVPlan.pdf}{VnV Plan}. If a functional requirement was determined to be satisfied or not staisfied by the tests with a "*" flag, then more explanation is given below:

\begin{itemize}
    \item FR1.3: UI components and pages are implemented to allow users and admins to edit submitted requests, however, work must now be done to connect backend to frontend UI to allow for changes to be reflected in the database.
    \item FR1.5: The system does not currently notify users of changes to the status of their reimbursement request, however, the email notification service/module is set up and can be executed in an isolated environment. Work must now be done ensure the email service can dynamically email users with updates for their submitted requests.
    \item FR1.6: UI is implemented to allow users to select a budget category. Budget categories need to be dynamically pulled from a database collection. Work must be done to connect backend service (Plaid API) to frontend UI to allow budget categories to be linked to requests and reflected in the database.
    \item FR1.7: Due to the fact that authentication is not completely implemented, this requirement cannot be satisfied. We do have page routing and UI implemented to distinguish admin page views and user page views.
    \item FR3.2: Authentication is not fully set up. Currently, we are manually setting user browser tokens when creating a user and logging in, work must be done to either dynamically assign user browser token, or adjust the authentication process to user vanilla username (email) and password for the login process.
\end{itemize}

\begin{table}[ht]
    \centering
    \footnotesize
    \caption{Functional Requirements Evaluation}
    \label{tab:FReval}
    \begin{tabular}{|p{1.2cm}|p{2.7cm}|p{2.4cm}|p{2.2cm}|p{1.4cm}|}
    \hline
    \textbf{FR} & \textbf{Modules} & \textbf{Test Suite} & \textbf{Test Cases} & \textbf{Satisfied (Y/N)} \\
    \hline
    FR1.1 & M3, M5, M9, M10, M11, M12, M13 & reconciler.test.js & FR1.1-TC1, FR1.1-TC2 & Yes \\
    \hline
    FR1.2 & M3, M5, M9, M10, M11, M12, M13 & requests.test.js & FR1.2-TC1, FR1.2-TC2 & Yes\\
    \hline
    FR1.3 & M3, M5, M9, M10, M11, M12, M13 & requests.test.js & FR1.3-TC1 & Yes*\\
    \hline
    FR1.4 & M3, M5, M9, M10, M11, M12, M13 & reconciler.test.js & FR1.4-TC1 & Yes\\
    \hline
    FR1.5 & M4, M7 & emailer.test.js & FR1.5-TC1 & No*\\
    \hline
    FR1.6 & M12 & reconciler.test.js & FR1.6-TC1 & No*\\
    \hline
    FR1.7 & M6 & Acct. mgmt. TS (TBD) & FR1.7-TC1, FR1.7-TC2 & No*\\
    \hline
    FR1.8 & M5, M3, M9, M13 & requests.test.js & FR1.8-TC1 & No\\
    \hline
    FR1.9 & M12 & reconciler.test.js & FR1.9-TC1 & No\\
    \hline
    FR3.1 & M10, M11 & Acct. mgmt. TS (TBD) & FR3.1-TC1 & Yes\\
    \hline
    FR3.2 & M6 & emailer.test.js, Acct. mgmt. TS (TBD) & FR3.2-TC1, FR3.2-TC2 & No*\\
    \hline
    FR3.3 & M2, M8 & & FR3.3-TC1 & Yes\\
    \hline
    \end{tabular}
    \normalsize
\end{table}

\newpage

\section{Nonfunctional Requirements Evaluation}

\subsection{Usability}
		
\subsection{Performance}

\subsection{etc.}
	
\section{Comparison to Existing Implementation}	

This section will not be appropriate for every project.

\section{Unit Testing}

Most files, excluding the 3rd-party library, top-level and GUI modules had corresponding unit tests.
All tests were performed using jest for testing javascript, and each commit to the main branch must pass all the tests.
All the tests passed as seen in the Test Report \ref{test_report}.

\section{Changes Due to Testing}

\wss{This section should highlight how feedback from the users and from 
the supervisor (when one exists) shaped the final product.  In particular 
the feedback from the Rev 0 demo to the supervisor (or to potential users) 
should be highlighted.}

\section{Automated Testing}

The tests were set up to automatically run in GitHub Actions whenever a commit was pushed to the $\mathtt{main}$ branch.
The configuration for the CI/CD automation can be found at \url{}
and the test pipeline can be viewed at \url{}.
		
\section{Trace to Requirements}
\begin{table}[ht]
  \caption{Trace from Test to Requirements}
  \centering
  \begin{tabular}{c c}
    \hline
    Test Suite & Requirement(s) \\
    \hline
    emailer.test.js & FR1.5, FR3.2 \\
    reconciler.test.js & FR1.1, FR1.4, FR1.6, FR1.9, INR1, SPLR1-3 \\
    requests.test.js & FR1.2, FR1.3, FR1.8, PVR1 \\ 
    Account management test suite (TBD) & FR1.7, FR3.1, FR3.2 \\
    Usability testing (via checklist) & APR1, APR2, STYR1, STYR2, EUR1-3, PIR1, LER2, UAPR1, ACSR1, SCR1-5, CLTR1 \\
    Static analysis (via checklist) & POAR1, LOR1, WR1, RLR1, ACSR2, MR2, SR1, LR1, LR2, STR1 \\
    Load testing & CPR1-3, SER2 \\
    \hline
  \end{tabular}
\end{table}
		
\section{Trace to Modules}		
\begin{table}[ht]
  \caption{Trace from Test to Modules}
  \centering
  \begin{tabular}{c c}
    \hline
    Test Suite & Module(s) \\
    \hline
    emailer.test.js & Notification Module (M4), Authentication Module (M6), Emailer API (M7) \\
    reconciler.test.js & Clubs, User, Request Database(M10-12) \\
    requests.test.js & Requests Module(M3), Requests Controller (M9) \\
    Account management test suite (TBD) & Account Management Module (M2), User Dashboard Module (M5), Account Management Controller (M8) \\
    Usability testing (via checklist) & Graphical User Interface (M13) \\    
    
  \end{tabular}
\end{table}

\section{Code Coverage Metrics}

\section{Appendix}
\subsection{Test Report} \label{test_report}
\begin{small} 
  \begin{verbatim} 
     PASS  tests/requests.test.js
      Request Endpoints and Model Validations
        Model Validations
          should require necessary reimbursement fields (3 ms)
        Reimbursement Endpoints
          should create reimbursement with file upload (34 ms)

    Test Suites: 1 passed, 1 total
    Tests:       2 passed, 2 total
    Snapshots:   0 total
    Time:        0.927 s, estimated 1 s
    Ran all test suites.
  \end{verbatim}
\end{small}

\bibliographystyle{plainnat}
\bibliography{../../refs/References}

\newpage{}
\section*{Appendix --- Reflection}

The information in this section will be used to evaluate the team members on the
graduate attribute of Reflection.

\input{../Reflection.tex}

\begin{enumerate}
  \item What went well while writing this deliverable? 
  \item What pain points did you experience during this deliverable, and how
    did you resolve them?
  \item Which parts of this document stemmed from speaking to your client(s) or
  a proxy (e.g. your peers)? Which ones were not, and why?
  \item In what ways was the Verification and Validation (VnV) Plan different
  from the activities that were actually conducted for VnV?  If there were
  differences, what changes required the modification in the plan?  Why did
  these changes occur?  Would you be able to anticipate these changes in future
  projects?  If there weren't any differences, how was your team able to clearly
  predict a feasible amount of effort and the right tasks needed to build the
  evidence that demonstrates the required quality?  (It is expected that most
  teams will have had to deviate from their original VnV Plan.)
\end{enumerate}

\end{document}
