\documentclass{article}

\usepackage{float}
\restylefloat{table}

\usepackage{booktabs}

\title{Team Contributions: Rev 0\\\progname}

\author{\authname}

\date{}

\input{../Comments}
%% Common Parts

\newcommand{\progname}{McMaster Engineering Society Custom Financial
Expense Reporting Platform} % PUT YOUR PROGRAM NAME HERE
\newcommand{\authname}{Team \#12, Reimbursement Rangers
\\ Adam Podolak
\\ Evan Sturmey
\\ Christian Petricca
\\ Austin Bennett
\\ Jacob Kish} % AUTHOR NAMES                  

\usepackage{hyperref}
    \hypersetup{colorlinks=true, linkcolor=blue, citecolor=blue, filecolor=blue,
                urlcolor=blue, unicode=false}
    \urlstyle{same}
                                


\begin{document}

\maketitle

This document summarizes the contributions of each team member for the Rev 0
Demo.  The time period of interest is the time between the POC demo and the Rev
0 demo. (November 14th, 2024 - February 11th, 2025)

\section{Demo Plans}

During the demonstration, we will showcase the fully finished system, highlighting its core functionality and user workflows. This will include demonstrating how users can seamlessly access and utilize the various request systems, such as submitting, reviewing, and processing financial requests. We will walk through the platform's key features, emphasizing its modular structure, backend functionality, and user-friendly interface. Additionally, we will illustrate how the system effectively handles error validation, data storage, and streamlined communication between different modules to ensure a smooth and efficient user experience. This demonstration will reflect the system’s readiness for deployment and its ability to meet the specified requirements.\\ 

\wss{What will you be demonstrating}

\section{Team Meeting Attendance}

\wss{For each team member how many team meetings have they attended over the
time period of interest.  This number should be determined from the meeting
issues in the team's repo.  The first entry in the table should be the total
number of team meetings held by the team.}

\begin{table}[H]
\centering
\begin{tabular}{ll}
\toprule
\textbf{Student} & \textbf{Meetings}\\
\midrule
Total & 6\\
Adam Podolak & 6\\
Evan Sturmey & 6\\
Austin Bennett & 6\\
Christian Petricca & 6\\
Jacob Kish & 5\\
\bottomrule
\end{tabular}
\end{table}

\wss{If needed, an explanation for the counts can be provided here.}

\section{Supervisor/Stakeholder Meeting Attendance}

\wss{For each team member how many supervisor/stakeholder team meetings have
they attended over the time period of interest.  This number should be determined
from the supervisor meeting issues in the team's repo.  The first entry in the
table should be the total number of supervisor and team meetings held by the
team.  If there is no supervisor, there will usually be meetings with
stakeholders (potential users) that can serve a similar purpose.}

\begin{table}[H]
\centering
\begin{tabular}{ll}
\toprule
\textbf{Student} & \textbf{Meetings}\\
\midrule
Total & 2\\
Adam Podolak & 2\\
Evan Sturmey & 2\\
Austin Bennett & 2\\
Christian Petricca & 2\\
Jacob Kish & 2\\
\bottomrule
\end{tabular}
\end{table}

\wss{If needed, an explanation for the counts can be provided here.}

\section{Lecture Attendance}

\wss{For each team member how many lectures have they attended over the time
period of interest.  This number should be determined from the lecture issues in
the team's repo.  The first entry in the table should be the total number of
lectures since the beginning of the term.}

\begin{table}[H]
\centering
\begin{tabular}{ll}
\toprule
\textbf{Student} & \textbf{Lectures}\\
\midrule
Total & 2\\
Adam Podolak & 2\\
Evan Sturmey & 1\\
Austin Bennett & 2\\
Christian Petricca & 0\\
Jacob Kish & 0\\
\bottomrule
\end{tabular}
\end{table}

\wss{If needed, an explanation for the lecture attendance can be provided here.}

\section{TA Document Discussion Attendance}

\wss{For each team member how many of the informal document discussion meetings
with the TA were attended over the time period of interest.}

\begin{table}[H]
\centering
\begin{tabular}{ll}
\toprule
\textbf{Student} & \textbf{Lectures}\\
\midrule
Total & 1\\
Adam Podolak & 1\\
Evan Sturmey & 1\\
Austin Bennett & 1\\
Christian Petricca & 1\\
Jacob Kish & 1\\
\bottomrule
\end{tabular}
\end{table}

\wss{If needed, an explanation for the attendance can be provided here.}

\section{Commits}

\wss{For each team member how many commits to the main branch have been made
over the time period of interest.  The total is the total number of commits for
the entire team since the beginning of the term.  The percentage is the
percentage of the total commits made by each team member.}

\begin{table}[H]
\centering
\begin{tabular}{lll}
\toprule
\textbf{Student} & \textbf{Commits} & \textbf{Percent}\\
\midrule
Total & 249 & 100\% \\
Adam Podolak & 62 & 25\% \\
Evan Sturmey & 50 & 20\% \\
Austin Bennett & 86 & 34\% \\
Christian Petricca & 31 & 13\% \\
Jacob Kish & 20 & 8\% \\
\bottomrule
\end{tabular}
\end{table}

\wss{If needed, an explanation for the counts can be provided here.  For
instance, if a team member has more commits to unmerged branches, these numbers
can be provided here.  If multiple people contribute to a commit, git allows for
multi-author commits.}

\section{Issue Tracker}

\wss{For each team member how many issues have they authored (including open and
closed issues (O+C)) and how many have they been assigned (only counting closed
issues (C only)) over the time period of interest.}

\begin{table}[H]
\centering
\begin{tabular}{lll}
\toprule
\textbf{Student} & \textbf{Authored (O+C)} & \textbf{Assigned (C only)}\\
\midrule
Adam Podolak & 32 & 13 \\
Evan Sturmey & 6 & 10 \\
Austin Bennett & 26 & 10 \\
Christian Petricca & 0 & 9 \\
Jacob Kish & 0 & 10 \\
\bottomrule
\end{tabular}
\end{table}

\wss{If needed, an explanation for the counts can be provided here.}

\section{CICD}

For the reimbursement platform, CICD will streamline development by automating testing, integration, and deployment processes. GitHub Actions will serve as the foundation for this CICD pipeline. Whenever code changes are pushed or a pull request is made, GitHub Actions will trigger workflows to automatically run tests, build the application, and verify code quality.

Through this automated testing and validation, we can ensure new code does not introduce bugs, enhancing reliability. After successful tests, GitHub Actions can deploy updates to a staging environment, and eventually to production. This automated cycle allows the team to focus on feature development, reduces manual work, and ensures consistent and reliable deployments, resulting in a faster, more stable development workflow for the platform.

\wss{Say how CICD is used in your project}

\wss{If your team has additional metrics of productivity, please feel free to
add them to this report.}

\end{document}